\documentclass[pdftex,11pt,a4paper,notitlepage]{report}
\usepackage{fullpage}
\usepackage[utf8]{inputenc}
\usepackage{amsmath}
\usepackage{amsfonts}
\usepackage{amssymb}
\usepackage{url}
\usepackage[pdftex]{graphicx}
\usepackage[ngerman]{babel}
\newcommand{\HRule}{\rule{\linewidth}{0.5mm}}
\newcommand{\HRuleThin}{\rule{\linewidth}{0.1mm}}
\author{Matthias Hecker}
\pagestyle{empty}
\begin{document}

% Header
\begin{center}
{ \large \bfseries Rails Weeks 2011}\\[0.4cm]
{ \huge \bfseries Rbot Plugin Community}\\[0.1cm]
\HRule \\[0.0cm]
\end{center}

\begin{flushleft}

\textbf{Autoren:}\hfill\parbox[t]{0.7\textwidth}{\raggedleft \today}

Matthias Hecker

Ivo Senner

Philipp Promeuschel\\[0.3cm]

\HRule \\[0.5cm]

\end{flushleft}

\begin{flushleft}
\textbf{Was ist Rbot?}
\end{flushleft}

\textit{Rbot}\footnote{\url{http://ruby-rbot.org/}} ist ein IRC-Bot (\textit{Internet Relay Chat}), 
vergleichbar mit dem bekannteren
\textit{Eggdrop} und ebenfalls ein Freies Software Projekt (lizenziert unter der
\textit{GPLv3}), allerdings vollständig in Ruby entwickelt. IRC-Bots verbinden sich
wie gewöhnliche IRC Clients zum Netzwerk und stellen Chat Teilnehmern
bestimmte Funktionen zur Verfügung wie das Übersetzen von Texten, das Lösen
von Rechenaufgaben, das Anzeigen von Informationen aus Webseiten, die Benachrichtigung
von bestimmten Ereignissen oder das Erfüllen von administrativen Aufgaben.

\begin{flushleft}
\textbf{Projektidee}
\end{flushleft}

Der Bot kann sehr einfach durch kurze in Ruby geschriebene Plugins erweitert werden,
seit 2006 existiert eine Webseite\footnote{\url{http://rbot.noway.ratry.ru/}} zum Zentralen 
veröffentlichen dieser Plugins.
Leider ist die Entwicklung dieser Webseite seitdem stehen geblieben, und es fehlen
einige wünschenswerte Funktionen. Unsere Idee ist eine Neuentwicklung einer Community 
zwischen Rbot Plugin Autoren und Benutzern. Die Rails 3.1 Anwendung wird
unter der GPLv3 lizenziert auf Github 
veröffentlicht\footnote{\url{https://github.com/RailsWeek2011/rbot-plugin-community}\\
    Testinstallation: \url{http://rbot-plugins.sixserv.org/}}. 
 
Es ist geplant,
nachdem die Software stabil genug arbeitet, die alte Webseite zu ersetzen und alle
existierenden Plugins zu migrieren.

\begin{flushleft}
\textbf{Funktionalität}
\end{flushleft}

\begin{itemize}
\item \textit{Benutzerverwaltung} (Registrieren, Login, Logout, Passwort Zurücksetzen, Passwort/Email Ändern, Benutzer Löschen)
\item \textit{Plugins} (Auflisten, Durchsuchen, Erstellen, Ändern, Löschen)
\item \textit{Versionen/Dateien} (Hochladen, Ändern, Löschen)
\item \textit{Bewertungen} (Plugins Bewerten von 1 bis 5)
\item \textit{Kommentare} (Kommentieren von Plugins, Schutz vor Spam per \textit{Akismet\footnote{\url{http://akismet.com/}}})
\item \textit{Feeds}
\item \textit{eMail und IRC Benachrichtigungen}
\end{itemize}

\end{document}
